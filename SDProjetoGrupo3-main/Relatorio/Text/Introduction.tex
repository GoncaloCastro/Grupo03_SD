\par Atualmente, existem vários modelos e técnicas para desenvolver aplicações, existindo varias dificuldades como por exemplo o controlo das dependências de uma aplicação, ou a estabilidade da aplicação independentemente da maquina em que se encontra a ser executada.

\par Existe no entanto uma forma que permite atualizações das aplicações sem ser necessário cancelar a sua execução na totalidade (devido à modularidade), e a sua execução sem depender do sistema operativo da maquina em que se encontram a ser executadas, sendo isto possível a partir da conteinerização.

\par Neste projeto utiliza-se a plataforma Docker, de forma a criar a imagem da aplicação para posteriormente criar os contentores, sendo usado em conjunção com o Kubernetes, permitindo a criação e gestão de clusters.

\par Na fase 1 o fluxo de trabalho foi criado tendo em conta a utilização do Visual Studio Code, do Github Actions, do Azure Container Registry e do Azure Kubernetes Service.

\par No entanto, na fase 2 utilizou-se apenas o Visual Studio Code e o Github Actions, tendo sido adicionado o DockerHub, e os Raspberry Pi's que foram usados para a criação do cluster local.
