\par No âmbito da unidade curricular de Sistemas distribuídos foi-nos proposto a realização de um projeto usando a tecnologia Docker para publicar uma aplicação previamente criada e construir um fluxo de trabalho de integração e entrega contínua (CI/CD, Continuous Integration/Continuous Delivery). Numa primeira fase criou-se o fluxo de trabalho de integração e entrega contínua, e numa segunda fase criou-se um cluster local utilizando 2 raspberry pi's.

\par Para a criação deste fluxo de trabalho usou-se varias tecnologias e ferramentas. Para a alteração de código, relativo ao fluxo de trabalho, utilizou-se o Visual Studio Code, para a criação de contentores utilizou-se o Docker, para o repositório de código foi utilizado o Github, tendo sido usado o Github Actions para automatizar a criação da imagem a partir do Dockerfile e posterior publicação no Azure Container Registry, tendo este sido usado em conjunto com o Azure Kubernetes Service para criar os Clusters para correr a aplicação.

\par Referente à segunda parte do projeto, criou-se um cluster local a partir de Raspberry Pi (RPI) e K3s, sendo que relativamente as aplicações usadas para o fluxo de trabalho e entrega continua utilizou-se o Visual Studio Code, o Github Actions e o DockerHub.
