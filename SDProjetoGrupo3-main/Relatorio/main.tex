\documentclass[12pt,a4paper]{article}

% encoding
%--------------------------------------
\usepackage[T1]{fontenc}
\usepackage[utf8]{inputenc}
%--------------------------------------

% Portuguese-specific commands
%--------------------------------------
\usepackage[portuguese]{babel}
%--------------------------------------

% Hyphenation rules
%--------------------------------------
\usepackage{hyphenat}
\hyphenation{mate-mática recu-perar}
%--------------------------------------

%  Para colocar o código (Programação) e fazer o highlighting
% The outputdir option tells where the minted aux files should be put, or where they can be found.
%--------------------------------------
% \usepackage[outputdir=AuxFiles/]{minted}
%--------------------------------------
% Para múltiplas colunas
%--------------------------------------
\usepackage{multicol}
%--------------------------------------

% Para imagens
%--------------------------------------
\usepackage{graphicx}
\graphicspath{Images/}
%--------------------------------------

% Para a fonte ser Arial
%--------------------------------------
\usepackage{helvet}
\renewcommand{\familydefault}{\sfdefault}
%--------------------------------------

% Realiza a indentação do paragrafo seguinte à secção.
%--------------------------------------
\usepackage{indentfirst}
%--------------------------------------

% Usado para criar a bibliografia
%--------------------------------------
%\usepackage{biblatex}
%\addbibresource{Bibliografia.bib}
%--------------------------------------

% Para os URL
%Opção hidelinks faz com que não apareçam caixas a volta dos links
%--------------------------------------
\usepackage[hidelinks]{hyperref}
%--------------------------------------

% Adiciona a Bibliografia automática à tabela de conteúdos(Índice) e dá-lhe um numero
%--------------------------------------
%\usepackage[nottoc,numbib]{tocbibind}
%--------------------------------------

%Por usar o package biblatex é recomendado usar este "to ensure that quoted texts are typeset according to the rules of your main language"
%--------------------------------------
%\usepackage{csquotes}
%--------------------------------------

%recommended solution for setting spacing between paragraphs
%--------------------------------------
\usepackage{parskip}
\setlength{\parindent}{3ex}% indica para dar um indent com 3ex no inicio do paragrafo
%--------------------------------------

%
%--------------------------------------
\usepackage{array}
%--------------------------------------

%--------------------------------------
\usepackage{tabularx}
%--------------------------------------

%Para o alinhamento do texto puder ser posto outra vez em justificado
%--------------------------------------
\usepackage{ragged2e}
%--------------------------------------

%
%--------------------------------------
% \usepackage{listings}
%--------------------------------------

%Para puder usar captions e labels com o minted
%--------------------------------------
\usepackage[newfloat, outputdir=AuxFiles/]{minted}
\usepackage{caption}

\newenvironment{code}{\captionsetup{type=listing}}{}
\SetupFloatingEnvironment{listing}{name=Source Code}
%--------------------------------------


%Used to set the line Spacing
%--------------------------------------
\renewcommand{\baselinestretch}{1.5}
%--------------------------------------


%Next line disables "1: Command terminated with space."
% chktex-file 1
%Next line disables "8: Wrong length of dash may have been used."
% chktex-file 8
%Next line disables "13: Intersentence spacing (`\@') should perhaps be used."
% chktex-file 13


\begin{document}

    \begin{titlepage}
        \centering

        \vspace{1cm}%Adiciona espaço vertical

        \par{% paragrafo
            \large %aumenta o tamanho da letra
            Faculdade das Ciências Exatas e da Engenharia
            \newline \normalsize Mestrado em Engenharia Informática
        }

        \begin{center}
            \begin{figure}[h]
                \includegraphics[width=13cm]{Images/UniversityLogo/LogoUniversidade.png}
            \end{figure}
        \end{center}

        \par{
            \Centering
            \normalsize Sistemas distribuídos
        }

        \vspace{1.5cm}

        \par{
            \Centering
            \large
            Projeto Prático - Fluxo de trabalho de CI/CD para aplicações na nuvem com AKS e K3s
        }

        \vspace{3cm}

        \begin{multicols}{2} %Coloca duas colunas

            \raggedright

                \par{
                    \normalsize
                    \textbf{Docentes:} \newline
                        \-\hspace{3ex}Docente: Karolina Baras \newline

                }

                \columnbreak

                \textbf{Trabalho Realizado por:} \newline
                    \-\hspace{3ex}Gonçalo de Castro - 2084515\newline
                    \-\hspace{3ex}José Sousa - 2027617\newline
                    \-\hspace{3ex}Rúben Silva - 2037517\newline
        \end{multicols}

        \vfill  %Preenche o espaço em branco

    % Bottom of the page

        \par{\small
        Funchal, \today
        } %Local e Data

    \end{titlepage}

    \newpage

    \renewcommand{\contentsname}{Índice} %Altera o nome a tabela de conteúdos para índice
    \tableofcontents %cria uma tabela de conteúdos

    \newpage

    \justify

    \section{Resumo}
        \par No âmbito da unidade curricular de Sistemas distribuídos foi-nos proposto a realização de um projeto usando a tecnologia Docker para publicar uma aplicação previamente criada e construir um fluxo de trabalho de integração e entrega contínua (CI/CD, Continuous Integration/Continuous Delivery). Numa primeira fase criou-se o fluxo de trabalho de integração e entrega contínua, e numa segunda fase criou-se um cluster local utilizando 2 raspberry pi's.

\par Para a criação deste fluxo de trabalho usou-se varias tecnologias e ferramentas. Para a alteração de código, relativo ao fluxo de trabalho, utilizou-se o Visual Studio Code, para a criação de contentores utilizou-se o Docker, para o repositório de código foi utilizado o Github, tendo sido usado o Github Actions para automatizar a criação da imagem a partir do Dockerfile e posterior publicação no Azure Container Registry, tendo este sido usado em conjunto com o Azure Kubernetes Service para criar os Clusters para correr a aplicação.

\par Referente à segunda parte do projeto, criou-se um cluster local a partir de Raspberry Pi (RPI) e K3s, sendo que relativamente as aplicações usadas para o fluxo de trabalho e entrega continua utilizou-se o Visual Studio Code, o Github Actions e o DockerHub.


    \newpage

    \section{Introdução}
        \par Atualmente, existem vários modelos e técnicas para desenvolver aplicações, existindo varias dificuldades como por exemplo o controlo das dependências de uma aplicação, ou a estabilidade da aplicação independentemente da maquina em que se encontra a ser executada.

\par Existe no entanto uma forma que permite atualizações das aplicações sem ser necessário cancelar a sua execução na totalidade (devido à modularidade), e a sua execução sem depender do sistema operativo da maquina em que se encontram a ser executadas, sendo isto possível a partir da conteinerização.

\par Neste projeto utiliza-se a plataforma Docker, de forma a criar a imagem da aplicação para posteriormente criar os contentores, sendo usado em conjunção com o Kubernetes, permitindo a criação e gestão de clusters.

\par Na fase 1 o fluxo de trabalho foi criado tendo em conta a utilização do Visual Studio Code, do Github Actions, do Azure Container Registry e do Azure Kubernetes Service.

\par No entanto, na fase 2 utilizou-se apenas o Visual Studio Code e o Github Actions, tendo sido adicionado o DockerHub, e os Raspberry Pi's que foram usados para a criação do cluster local.


    \newpage

    \section{Descrição das atividades}
        \subsection{Fase 1}

    \par Nesta fase foi realizado todo o processo base do fluxo de trabalho, tendo sido escolhida a aplicação de teste (aplicação Todo App, utilizada durante a aula).

    \par No workflow (ver Source Code~\ref{code:workflow_fase_1}) definiu-se que quando for realizado um push para o branch \emph{main}, o fluxo de trabalho fica ativo, sendo depois criada a imagem da app e posteriormente é feito o push da imagem para o Azure Container Registry (ACR), para alem disso, as credenciais de acesso ao ACR foram colocadas como segredos do repositório.

    \begin{code}
        \inputminted[linenos,tabsize=1,breaklines,lastline=21]{yaml}{../.github/workflows/project-workflow-fase-1.yml}
        \caption{Workflow utilizado na Fase 1}
        \label{code:workflow_fase_1}
    \end{code}

    \par Relativamente ao Dockerfile, o codigo foi utilizado de um dockerfile fornecido numa das aulas da unidade curricular, tendo sido adicionada a linha 5 e tendo sido feito algumas alterações (ver Source Code~\ref{code:dockerfile}).

    \begin{code}
        \inputminted[linenos,tabsize=1,breaklines]{dockerfile}{../todo_app/app/Dockerfile}
        \caption{DockerFile da app}
        \label{code:dockerfile}
    \end{code}

    \par Relativamente ao fluxo de trabalho foram realizados os seguintes passos:
    \begin{enumerate}
        \item Criou-se um grupo de recursos:

        \begin{minted}[tabsize=1,breaklines]{shell}
az group create --name ProjetoSDGrupo3 --location eastus
        \end{minted}

        \item Criou-se um registo de contentores (Azure Container Registry) no Azure:

        \begin{minted}[tabsize=1,breaklines]{shell}
az acr create --resource-group ProjetoSDGrupo3 --name projetosdgrupo3 --sku Standard
        \end{minted}

        \item Ativou-se \emph{admin} no ACR:

        \begin{minted}[tabsize=1,breaklines]{shell}
az acr update -n projetosdgrupo3 --admin-enabled true
        \end{minted}

        \item Criou-se um Cluster do AKS:

        \begin{minted}[tabsize=1,breaklines]{shell}
az aks create --resource-group ProjetoSDGrupo3 --name projetosdgrupo3AKSCluster --node-count 2 --generate-ssh-keys
        \end{minted}

        \item Obteve-se as credenciais do Cluster AKS:

        \begin{minted}[tabsize=1,breaklines]{shell}
az aks get-credentials --resource-group ProjetoSDGrupo3 --name projetosdgrupo3AKSCluster
        \end{minted}

        \item Conectou-se o Cluster AKS ao ACR:

        \begin{minted}[tabsize=1,breaklines]{shell}
az aks update --attach-acr projetosdgrupo3 --name projetosdgrupo3AKSCluster --resource-group ProjetoSDGrupo3
        \end{minted}


        \item Obteve-se as credenciais de acesso ao ACR:

        \begin{minted}[tabsize=1,breaklines]{shell}
az acr show -n projetosdgrupo3 --query loginServer -o tsv
az acr credential show -n projetosdgrupo3 --query username -o tsv
az acr credential show -n projetosdgrupo3 --query passwords[0].value -o tsv
        \end{minted}

        \item Colocou-se as credencias do ACR como segredos no repositório.
        \item Fez-se um push para o \emph{main} de forma a ativar o workflow.
        \item No Azure Cli, criamos um ficheiro \emph{app.yaml}, e copiamos o código presente no ficheiro \emph{app.yaml} (ver Source Code~\ref{code:app_yaml}).

        \begin{minted}[tabsize=1,breaklines]{shell}
nano app.yaml
        \end{minted}

        \item Criou-se o cluster a partir do ficheiro \emph{app.yaml}.

        \begin{minted}[tabsize=1,breaklines]{shell}
kubectl apply -f app.yaml
        \end{minted}

        \item Obteve-se o endereço IP externo para abrir a app no browser.

        \begin{minted}[tabsize=1,breaklines]{shell}
kubectl get service todo-app --watch
        \end{minted}

    \end{enumerate}




\subsection{Fase 2}

    \par Nesta fase o procedimento realizado foi diferente do realizado na fase 1, não tendo sido usado o Azure Container Registry nem o Azure Kubernetes Service. Nesta fase utilizou-se o Github Actions e o DockerHub para o fluxo de trabalho e de integração continua.
    \par Para a criação do cluster local foram utilizados 2 raspberry pi's, sendo que um é o master e o outro é o worker/slave. Foram atribuidos 2 endereços IP, 1 para cada RPI (ver Tabela~\ref{table:IP_address_password_hostname_rpi})

    \begin{table}[h!]
        \centering

        \begin{tabular}{ c c c c }
            \hline
            \textbf{Role} & \textbf{IP Address} & \textbf{Password} & \textbf{Hostname} \\
            \hline
            Master & 10.2.15.156 & \emph{Node3\_2} & node32 \\
            % \hline
            Worker/Slave & 10.2.15.155 & \emph{Node3\_1} & node31 \\
            \hline
        \end{tabular}

        \caption{Endereços IP, Password e Hostname dos Raspberry Pi's.}

        \label{table:IP_address_password_hostname_rpi}
    \end{table}

    \par Para a configuração dos RPI, utilizou-se 2 cartões micro sd preparados anteriormente para possibilitar o acesso aos RPI's a partir de ssh, e para poder iniciar os RPI's a partir de uma pen USB.
    \par Para a configuração das pens USB, a serem usadas para iniciar cada um dos RPI's, foram realizados os seguintes passos:

    \begin{enumerate}
        \item Gravou-se o sistema operativo (Raspbian OS Lite 32-bits).
        \item Criou-se um ficheiro ssh no diretório boot do sistema.
        \item Editou-se o ficheiro 'cmdline.txt' adicionando o seguinte separado por espaços:

        \begin{itemize}
            \item No ficheiro do worker/slave:

            \begin{itemize}
                \item cgroup\_enable \( = \) cpuset
                \item cgroup\_memory \( = \) 1
                \item cgroup\_enable \( = \) memory
                \item ip \( = \) 10.2.15.155
            \end{itemize}

            \item No ficheiro do master:

            \begin{itemize}
                \item cgroup\_enable \( = \) cpuset
                \item cgroup\_memory \( = \) 1
                \item cgroup\_enable \( = \) memory
                \item ip \( = \) 10.2.15.156
            \end{itemize}

        \end{itemize}

        \item Ligou-se cada pen USB ao respetivo RPI em substituição do cartão SD.
        \item Alterou-se as configurações do RPI (Ativar SSH; mudar a password, alterar o hostname. Ver Tabela \ref{table:IP_address_password_hostname_rpi}).
        \item Editou-se os ficheiros '/etc/dhcpcd.conf' e 'cmdline.txt', retirando comentarios e adicionando o IP, e apagando o IP de cada ficheiro respetivamente.
        \item Trocou-se a framework \textbf{nftables} pela \textbf{iptables}. Realizando depois um reboot dos RPI's.
        \item De seguida instalou-se e configurou-se o K3s com os seguintes comandos:

        \begin{enumerate}
            \item No Master:

            \begin{enumerate}
                \item \mintinline{shell}{K3S_KUBECONFIG_MODE="644"}
                \item \mintinline{shell}{curl -sfL https://get.k3s.io | sh -}
                \item Testou-se se a instalação ocorreu sem erros usando:\\
                    \mintinline{shell}{sudo kubectl get nodes}
                \item Copiou-se o Token do Master:\\
                    \mintinline{shell}{sudo cat /var/lib/rancher/k3s/server/node-token}
            \end{enumerate}

            \item No Worker/Slave:

            \begin{enumerate}
                \item Conectou-se o Worker/Slave com o Master, usando o IP do Master e o Token copiado:

                \begin{minted}[breaklines,tabsize=1]{shell}
curl -sfL http://get.k3s.io | K3S_URL=https://IP_DO_MASTER:6443 K3S_TOKEN=TOKEN_OBTIDO_PREVIAMENTE sh -
                \end{minted}

            \end{enumerate}
        \end{enumerate}

        \item Testou-se no Master se a conexão com o Worker/Slave funciona:\\
            \mintinline{shell}{sudo kubectl get nodes}

    \end{enumerate}

    \par Relativamente à criação da imagem, criou-se um workflow do github relativo à esta fase, para que quando seja realizado um push para o branch \emph{main}, seja criada a imagem e colocada no DockerHub (ver Source Code~\ref{code:buid_image}).

        \begin{code}
            \inputminted[linenos,breaklines,tabsize=1,firstline=19,lastline=24]{yaml}{../.github/workflows/project-workflow-fase-2.yml}

            \captionof{listing}{Image build}
            \label{code:buid_image}
        \end{code}

    \par Após a imagem ter sido colocada no DockerHub, e os RPI's terem a configuração do K3s concluída e a funcionar, utilizou-se o ficheiro \emph{app.yaml} (ver Source Code~\ref{code:app_yaml}) para criar o cluster.


    \newpage

    \section{Análise de resultados}
        % TODO: Análise dos resultados 2 a 4 Paginas




\par Análise dos resultados 2 a 4 Paginas


    \newpage


    \section{Conclusão e trabalho futuro}
        % TODO: Conclusão e trabalho futuro 1 a 2 Paginas

\par Com este trabalho, colocou-se em prática e aprofundou-se os conceitos lecionados na unidade curricular, nomeadamente os conceitos relativos à conteinerização, assim como os relativos à utilização do Docker, do Kubernetes, do Azure DevOps e do K3s e Raspberry Pi.

\par Obteve-se também o conhecimento básico relativo à utilização do Git-hub Actions que nos permitiu a automatização do fluxo de trabalho.

\par Relativamente ao trabalho futuro, considera-se importante a automatização total do fluxo de trabalho a partir do momento em que é efetuada uma alteração no repositório (assim que for realizado um push ou um pull request), de forma que o Workflow criado, realize todos os passos necessários tendo em conta se é necessário parar a execução da aplicação ou não para ser realizada a correção ou atualização de um ou mais modulos dessa aplicação.


    \newpage


    \section{Anexos}
        
\begin{code}
    \inputminted[linenos,breaklines,tabsize=1,]{yaml}{../todo_app/app/app.yaml}
    \captionof{listing}{Ficheiro descrição do cluster.}
    \label{code:app_yaml}
\end{code}


\end{document}
